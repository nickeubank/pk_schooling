% So we make this "beamer" rather than document! 

\documentclass{beamer}
\usetheme{default}
\usecolortheme{beaver}
\setbeamertemplate{navigation symbols}{}

\usepackage{beamerfoils}
%\usepackage{natbib} 
\usepackage[natbib=true, bibstyle=authoryear, citestyle=authoryear-comp]{biblatex}
\usepackage{graphicx} 
%\usepackage{geometry}
\bibliography{/Users/Nick/Dropbox/Dropbox_Documents/my_library}
%\bibliographystyle{apalike}


%\setbeameroption{show notes}
%\setbeamertemplate{note page}[plain]

\title{Understanding Rural Private School Performance}
\author{Nick Eubank}
\date{\today}


% This is the beginning of a real document!
\begin{document} 

\begin{frame}{}
	\titlepage
\end{frame}

\begin{frame}{Context}
	\begin{enumerate}
		\item \textbf{Massive growth in rural private enrollment. }
			\begin{itemize}
				\item \cite{Andrabi:2008ji, Pratham:2005vw}
			\end{itemize}
		\pause
		\item \textbf{Private schools outperform government schools.}
			\begin{itemize}
				\item \cite{Jimenez:1991wa, Jimenez:1995vg, Pratham:2005vw, Andrabi:2011hl, Desai:2009ty, Tooley:2003vf, Alderman:2003we, Alderman:2001wk}
		\end{itemize}	
		\pause
		\item \textbf{Big push for voucher programs.}
		\begin{itemize}
			\item \cite{Chakrabarti:2008vc, Kelkar:2006tq, Panagariya:2008wi}
		\end{itemize}
	\end{enumerate}
\pause
But \emph{why} are private schools better?
\note{Enrollment in Pak increased by 47\% from 2000 to 2005. In 2005, was 1/3. In India, greater than 20\%.}

\end{frame}


\begin{frame}{Explanation 1: Teaching Quality}

Possibility 1: Better Inputs
\begin{itemize}
	\pause
	\item No
\end{itemize} 
\pause	

Possibility 2: Better Induced Effort
	\begin{itemize}
		\item Pay for performance, fire bad teachers.
		\pause
		\item Clearly improvement over public schools. 
		\begin{itemize}
			\item \cite{Muralidharan:2008tb, Chaudhury:2006vp}
		\end{itemize}
		\pause
		\item Importance shown in US research.
		\begin{itemize}
			\item \cite{Hanushek:1997tt,Hanushek:2003hz,Banerjee:2007wx}
		\end{itemize}
	\end{itemize}
\end{frame}
\pause
``Teacher Quality'' Story


\begin{frame}{Explanation 2: Sorting}
Maybe private school students are ``different.''
\pause
	\begin{itemize}
			\item $>$20\%  send one child to a private school and one child to a government school.
			\pause
		\item Some attempts to control through randomization of vouchers
			\begin{itemize}
				\item \cite{Angrist:2002up, Bellei:2008uu}
			\end{itemize}
		\pause
		\item But lots of problems...
			\begin{itemize}
				\item Risk of losing vouchers induces efforts
				\item Selective admission
			\end{itemize}
	\end{itemize}	

\pause
If true, then private school superiority is illusory. 
\begin{itemize}
	\item Vouchers could result in massive mis-allocation of resources.
\end{itemize}
\end{frame}


\begin{frame}{This Paper}
\pause	
Private school dominance declines by 50\% in fractionalized villages
	\pause
\begin{enumerate}
	\item This does \emph{not} arise because of changes in ``Teaching Quality''
	\pause
	\item This \emph{does} arise because of changes in ``Sorting.''
	\pause
		\begin{itemize}
			\item In homogeneous villages, school choice is based on academic potential.
			\pause
			\item In fractionalized villages, school choice is based on caste politics.
		\end{itemize}
\end{enumerate}
\pause
Tells us that \emph{at least} half of private school premium comes from selective sorting, not better teaching.
\end{frame}



\begin{frame}{}
	\begin{figure}[htb]
		\begin{center}
		\includegraphics[scale=0.4]{maps/hh_map_allpak.pdf}
		\end{center}
	\end{figure}
\end{frame}


\begin{frame}{}
	\begin{figure}[htb]
		\begin{center}
		\includegraphics[scale=0.4]{maps/concentrated.pdf}
		\end{center}
	\end{figure}
\end{frame}

\begin{frame}{}
	\begin{figure}[htb]
		\begin{center}
		\includegraphics[scale=0.4]{maps/disparate_farming.pdf}
		\end{center}
	\end{figure}
\end{frame}



\section{Fractionalization and Performance}\label{}
\begin{frame}{Outline}
	\tableofcontents[currentsection]
\end{frame}

% \begin{frame}{Caste in Punjab}
% \begin{itemize}
% 	\item Very similar to caste in India
% 	\begin{itemize}
% 		\item Biraderi implies ``an inherent, inbuilt hierarchy that governs social interactions''\citep[p. 29]{Gazdar:2007vt}.	
% 	\end{itemize}
% 	\item Not synonymous with economic status:
% 	\begin{itemize}
% 		\item ``the poorest Jatt is still better off than the richest kammi.'' \citep[p. 13]{Gazdar:2007vt} 
% 	\end{itemize}
% \end{itemize}
% \end{frame}
	


\begin{frame}{}
	\begin{figure}[h]
		\centering	
		\includegraphics[scale=0.8]{graphs/kids_combined.pdf}
	\end{figure}
\end{frame}

\begin{frame}{}
	\begin{figure}[htb]
		\begin{center}
		\includegraphics[scale=0.8]{graphs/kids_combined_district.pdf}
		\end{center}
	\end{figure}
	
\end{frame}





\begin{frame}{Conclusion}
Take-aways:
	\begin{enumerate}
		\item \emph{At least} 50\% of private school premium due to sorting.
		\pause
		\item Studying village characteristics can provide new insights into role of sorting.
	\end{enumerate}
\pause
Next Steps:
\begin{itemize}
	\item Develop more robust measures of social status.
	\item Test in India
\end{itemize}

\end{frame}
\end{document}