% So we make this "beamer" rather than document! 

\documentclass{beamer}
\usetheme{default}
\usecolortheme{beaver}
\setbeamertemplate{navigation symbols}{}

\usepackage{beamerfoils}
%\usepackage{natbib} 
\usepackage[natbib=true, bibstyle=authoryear, citestyle=authoryear-comp]{biblatex}
\usepackage{graphicx} 
%\usepackage{geometry}
\bibliography{/Users/Nick/Dropbox/Dropbox_Documents/my_library}
%\bibliographystyle{apalike}


%\setbeameroption{show notes}
%\setbeamertemplate{note page}[plain]

\title{Understanding Private School Performance in Rural Pakistan}
\author{Nick Eubank}
\date{\today}


% This is the beginning of a real document!
\begin{document} 

\begin{frame}{}
	\titlepage
\end{frame}

\begin{frame}{Context}
Explosion of Rural Private Schools
	\begin{itemize}
		\item Pakistan: from 2000 to 2005, enrollment increased 47\%. \\ 
			By 2005, 1/3 of students in private school.
		\item India: in 2005, $>$20\% of rural students in private school.
	\end{itemize}
\pause
Private schools radically outperform government schools.
	\begin{itemize}
		\item \citep{Jimenez:1991wa, Jimenez:1995vg, Pratham:2005vw, Andrabi:2011hl, Desai:2009ty, Tooley:2003vf, Alderman:2003we, Alderman:2001wk}
	\end{itemize}	
\pause
But why?
\end{frame}


\begin{frame}{Explanation 1: Better Teaching}

Better Inputs?
\begin{itemize}
	\pause
	\item No
\end{itemize}
\pause	
But perhaps better induced efforts
	\begin{itemize}
		\item Clearly problem in government:\citep{Muralidharan:2008tb, Chaudhury:2006vp}
		\item Consistent with US research: \cite{Hanushek:1997tt,Hanushek:2003hz,Banerjee:2007wx}
	\end{itemize}
\end{frame}


\begin{frame}{Explanation 2: Sorting}
	\begin{itemize}
		\item Even after controlling for HH wealth and parental education, maybe private school students are ``different.''
		\pause
		\begin{itemize}
			\item $>$20\% of households send send children to \emph{both} government and private schools. 
		\end{itemize}
		\item Some attempts to control through randomization of vouchers
			\begin{itemize}
				\item \citep{Angrist:2002up, Bellei:2008uu}
			\end{itemize}
		\pause
		\item But lots of problems...
			\begin{itemize}
				\item Risk of losing vouchers induces efforts
				\item Selective admission
			\end{itemize}
	\end{itemize}	

\pause
If true, then private school superiority is illusory. 
\begin{itemize}
	\item Problem in light of push for vouchers. \\
	\citep{Chakrabarti:2008vc, Kelkar:2006tq, Panagariya:2008wi}
\end{itemize}
\end{frame}


\begin{frame}{Contribution}
	Examines compatibility with novel finding: \\
	Private school dominance declines by 50\% with village caste fractionalization.
	\pause
	\begin{itemize}
		\item Consistent with private school dominance arising from selective ``sorting''
	\end{itemize}
\end{frame}



\section{Methodology}\label{}
\begin{frame}{Outline}
	\tableofcontents[currentsection]
\end{frame}

\begin{frame}{Data}	
Learning and Educational Attainment in Punjab Schools (LEAPS)
\begin{itemize}
	\item 2003-2007 panel data with data from teachers, students, households, and owners.
	\item One four year panel (12,110 children)
	\item One two year panel (11,852 children)
	\item 112 Villages
	\item Three Districts
	\item Includes: Child Test Scores, Teacher Test Scores, Parental Educational, HH Wealth
	\item Test scores are normalized using IRT -- mean 0, standard deviation 1.
\end{itemize}
\end{frame}

\begin{frame}{}
	\begin{figure}[htb]
		\begin{center}
		\includegraphics[scale=0.4]{maps/hh_map_allpak.pdf}
		\end{center}
	\end{figure}
\end{frame}

\begin{frame}{}
	\begin{figure}[htb]
		\begin{center}
		\includegraphics[scale=0.4]{maps/disparate_farming.pdf}
		\end{center}
	\end{figure}
\end{frame}

\begin{frame}{}
	\begin{figure}[htb]
		\begin{center}
		\includegraphics[scale=0.4]{maps/concentrated.pdf}
		\end{center}
	\end{figure}
\end{frame}


\begin{frame}{Measuring Learning}
	Lagged-Value-Added Model:	
	\begin{eqnarray}
		Y_{i,t}=\alpha_tX_{i,t}+\alpha_{t-1}X_{i,t-1}+ \dots + \alpha_1X_{i,1} + \epsilon_{i,t}
	\end{eqnarray}
	\pause
	\begin{eqnarray}
		Y_{i,t}=X_{i,t}\alpha+Y_{i,t-1}\beta + \epsilon_{i,t}\label{primary}
	\end{eqnarray}
	\begin{itemize}
		\item Flexible persistence parameter
		\item All past scores included to control of measurement error.
		\item Controls for differences in initial levels, but not differences in rates. 
	\end{itemize}
\end{frame}

\begin{frame}{Measuring Learning}
	Village Level:
	\begin{enumerate}
		\item Run Lagged-Value Added regressions with village-school type dummies for each village $j$.
		\begin{eqnarray*}
			Y_{i,t}=X_{i,t}\alpha+Y_{i,t-1}\beta + \mathbb{I}_{i,j,type,t}\gamma_{j, type}+\epsilon_{i,t}
		\end{eqnarray*}
		\item Extract dummies and calculate village public-private gap.
		\item Analyze at level of village.
	\end{enumerate}
	\begin{eqnarray*}
		Gap_{j}=Z_{j}\delta+\eta_{j}\label{villagespecification}
	\end{eqnarray*}
\end{frame} 

\begin{frame}{Measuring Learning}
	Teacher Level:
	\begin{enumerate}
		\item Run Lagged-Value Added regressions with teacher fixed effects dummies for each teacher $k$.
		\begin{eqnarray*}
			Y_{i,t}=X_{i,t}\alpha+Y_{i,t-1}\beta + \mathbb{I}_{i,k,t}\zeta_{k}+\epsilon_{i,t}\label{teacherspecification}
		\end{eqnarray*}
		
		\item Extract fixed effect coefficients as estimates of teacher contributions
		\item Analyze at level of teacher (weighted by number of students).
	\end{enumerate}
\end{frame}


\section{Fractionalization and Performance}\label{}
\begin{frame}{Outline}
	\tableofcontents[currentsection]
\end{frame}

\begin{frame}{Caste in Punjab}
\begin{itemize}
	\item Very similar to caste in India
	\begin{itemize}
		\item Biraderi implies ``an inherent, inbuilt hierarchy that governs social interactions''\citep[p. 29]{Gazdar:2007vt}.	
	\end{itemize}
	\item Not synonymous with economic status:
	\begin{itemize}
		\item ``the poorest Jatt is still better off than the richest kammi.'' \citep[p. 13]{Gazdar:2007vt} 
	\end{itemize}
\end{itemize}
\end{frame}
	
\begin{frame}{Caste in Punjab}
	\begin{figure}[htb]
		\begin{center}
		\includegraphics[scale=0.6]{graphs/village_frac_by_district.pdf}
		\end{center}
	\end{figure}		
\end{frame}

\begin{frame}{Caste in Punjab}
	% Note really correlated with stuff
\end{frame}

\begin{frame}{}
	\begin{figure}[h]
		\centering	
		\includegraphics[scale=0.8]{graphs/kids_combined.pdf}
	\end{figure}
\end{frame}

\begin{frame}{}
	\begin{figure}[htb]
		\begin{center}
		\includegraphics[scale=0.8]{graphs/kids_combined_district.pdf}
		\end{center}
	\end{figure}
	
\end{frame}

\section{School Quality}\label{}
\begin{frame}{Outline}
	\tableofcontents[currentsection]
\end{frame}

\begin{frame}{No Difference in Inputs}
	\begin{figure}[htb]
		\begin{center}
		\includegraphics[scale=0.5]{tables/Private_teacher_quality.pdf}
		\end{center}
	\end{figure}
\end{frame}

\begin{frame}{No Difference in Inputs}
	\begin{figure}[htb]
		\begin{center}
		\includegraphics[scale=0.5]{tables/govt_teacher_quality.pdf}
		\end{center}
	\end{figure}
	
\end{frame}

\begin{frame}{}
	\begin{figure}[htb]
		\begin{center}
		\includegraphics[scale=0.7]{graphs/absenteeism_and_pay.pdf} 
		\end{center}
	\end{figure}
\end{frame}

\begin{frame}{}
	\begin{figure}[htb]
		\begin{center}
			\includegraphics[scale=0.7]{graphs/compensation_scores.pdf}
		\end{center}
	\end{figure}
\end{frame}

\begin{frame}{No Difference in Incentives}
\begin{figure}[htb]
	\begin{center}
	\includegraphics[scale=0.7]{tables/frac_and_compensation.pdf}
	\end{center}
\end{figure}
\end{frame}

\section{Selective Sorting}\label{}
\begin{frame}{Outline}
	\tableofcontents[currentsection]
\end{frame}

\begin{frame}{Sorting}
	\begin{figure}[htb]
		\begin{center}
		\includegraphics[scale=0.65]{tables/mean_preserving.pdf}
		\end{center}
	\end{figure}
	
\end{frame}

\begin{frame}{A Sorting Story}
	\begin{description}
		\item [Homogenous Villages:] Children sort on academic potential.
		\item [Fractionalized Villages:] Children also sort by social status.
	\end{description}
	\pause
	\begin{enumerate}
		\item Parents pick winners
	\end{enumerate}
\end{frame}

\begin{frame}{}
	\begin{figure}[htb]
		\begin{center}
		\includegraphics[scale=0.7]{tables/intelligence_type.pdf}
		\end{center}
	\end{figure}
\end{frame}

\begin{frame}{}
	\begin{figure}[htb]
		\begin{center}
		\includegraphics[scale=0.4]{graphs/village_toptwo.pdf}
		\pause \\
		\includegraphics[scale=0.4]{graphs/school_toptwo.pdf}
		\end{center}
	\end{figure}
\end{frame}

\begin{frame}{}
	\begin{figure}[htb]
		\begin{center}		\includegraphics[scale=0.8]{graphs/intra_versus_intervillage_frac_combined.pdf}
		\end{center}
	\end{figure}
\end{frame}

\begin{frame}{Social Status and School Type}
	\begin{figure}[htb]
		\begin{center}
		\includegraphics[scale=0.5]{tables/social_status_type.pdf}
		\end{center}
	\end{figure}
\end{frame}

\begin{frame}{Fractionalization and Prices}
	\begin{figure}[htb]
		\begin{center}
		\includegraphics[scale=0.6]{tables/prices.pdf}
		\end{center}
	\end{figure}
\end{frame}

\begin{frame}{Inconsistencies}
\begin{figure}[htb]
	\begin{center}
	\includegraphics[scale=0.6]{tables/choice_interactions.pdf}
	\end{center}
\end{figure}

\end{frame}

\begin{frame}{Sorting Paradox}
Why pay more for the same education?
\pause
\begin{description}
	\item [Neighborhood Effects:] Students performance is affected by peers
	\pause
	\item [Networking:] About forming positive associations. 
	\begin{itemize}
		\item In homogenous villages, most important association is intelligence.
		\item In fractionalized villages, caste matters too. 
	\end{itemize}
\end{description}
\end{frame}


\section{Discussion}\label{}
\begin{frame}{Outline}
	\tableofcontents[current]
\end{frame}


\begin{frame}{}
	Places a lower bound on role of sorting. 
\end{frame}
\end{document}