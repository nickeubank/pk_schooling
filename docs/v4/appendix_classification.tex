\documentclass[Eubank_pk_ethnic_sorting.tex]{subfiles}



\begin{document}


Table~\ref{appendix_classification_table} shows classifications and sources. 

Land holdings are sufficient but necessary condition. 



\begin{table}[]
\centering
\caption{Social Status Classifications}
\label{appendix_classification_table}
\begin{tabular}{lcl}
\textbf{Biraderi}             & \textbf{Status} & \textbf{Notes} \\
Abbasi                        & High*  			& ``Section, both of [Syeds] and [Sheikhs]'' \citep[p. 341]{Blunt:1969vq}      \\
Ansari                        & Low*   			& ``[Sheikh] section''      \\
Arain                         & High*  			&       \\
Awan                          & High*  			&       \\
Baloch                        & Low*   			&       \\
Butt                          & High*  			&       \\
Charchar                      & ???    			& ``Kharwar'' are among ``Non-functional castes of low position'' \citep[p. 106]{Blunt:1969vq}. ?? \\
Dhobi / Naich / Mochi / Lohar & ???    			& Professional castes: laundry washers (Dhobi), blacksmiths (Lohar), cobblers (mochi) and Barbers? (if Naich is same as Nai)       \\
Gujjar                        & High  			& ``Non-functional castes of respectable position'' \citep[p. 106]{Blunt:1969vq} \\
Jat                           & High  			& ``Non-functional castes of respectable position'' \citep[p. 106]{Blunt:1969vq} \\
Kharar                        & ???    			&       \\
Lar                           & ???    			&       \\
Mohana                        & ???    			&       \\
Mughal                        & Low*   			&       \\
Muslim Sheikh                 & Low*   			&       \\
Non-Muslim					  & ???    			&       \\
Pathan                        & High  			& ``High caste'' \citep[p. 268]{Blunt:1969vq} \\
Qureshi / Hashmi              & Low*   			&  ``[Sheikh] section'' \citep[p. 353]{Blunt:1969vq}   \\
Rajput / Bhatti               & High  			& ``High caste'' \citep[p. 268]{Blunt:1969vq}; Warrior and land owning caste \citep[p. 353]{Blunt:1969vq}  \\
Rehmani                       & Low*   			&       \\
Samejha                       & ???    			&       \\
Sheikh                        & High  			& ``High caste'' \citep[p. 268]{Blunt:1969vq} \\
Solangi                       & Low*   			&       \\
Syed                          & High  			& ``High caste'' \citep[p. 268]{Blunt:1969vq}    
\end{tabular}
\end{table}


Note: Mohana in 15\_make\_hh\_child\_cross\_section; 

Kharar \& Charchar duplicates? \cite[p. 344]{Blunt:1969vq} says Charchar may be Charghar -- 4 houses. But doesn't say what it is besides caste subdivision.  

\cite[p. 10]{Blunt:1969vq} defines a ``Section'' or ``Sept.'' as the largest exogamous group within a ``subcaste'', which he defines as the smalled endogamous groups within a caste. 


Working with a local anthropologist, we constructed a caste-status identifier that categorizes dozens of distinctly named caste/clan (zaat/biradari) groups into “high” and ``low'' caste. High-caste includes all such groups that self-identify on the basis of traditional access to land (zamindars). The low-caste group comprises zaats that were historically considered either out-castes (similar to the dalits in India) or were in clientalist relationships with zamindars as providers of services in the village economy; i.e. barbers, metalworkers, clothes washers, etc. Based on this definition, around 25\% of the population from which we draw our sample consists of low-caste households, with the highest proportion (35\%) found in Sindh province.


\end{document}
