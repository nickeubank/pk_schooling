\documentclass[11pt]{article}

\usepackage{amsfonts, amsmath, amssymb}
\usepackage{dcolumn, multirow}
\usepackage{setspace}
\usepackage{epsfig, subfigure, subfloat, graphicx}
\usepackage{booktabs}
\usepackage{tabularx}
\usepackage{anysize, indentfirst, setspace}
\usepackage{verbatim, rotating, paralist}
\usepackage{pdfsync}
\usepackage{latexsym}
\usepackage{amsthm}
%\usepackage{fullpage}
\usepackage{longtable}
\usepackage{natbib}
\usepackage{graphicx}
\usepackage{mathabx}
\usepackage{txfonts}
\usepackage{amsfonts}
\usepackage{parskip}
\usepackage{booktabs}
\usepackage{stmaryrd}
\usepackage{mathrsfs}
\usepackage{dsfont}
\usepackage{comment}
\usepackage{url}
\usepackage{rotating}
\usepackage{hyperref}
\usepackage{appendix}
\usepackage{subfiles}

\usepackage[capposition=top]{floatrow}

\newcommand{\sym}[1]{\rlap{#1}}% Thanks to David Carlisle

% Allow line breaks with \\ in specialcells
	\newcommand{\specialcellc}[2][c]{\begin{tabular}[#1]{@{}c@{}}#2\end{tabular}}
	
	\newcommand{\specialcell}[2][c]{\begin{tabular}[#1]{@{}l@{}}#2\end{tabular}}




\usepackage[margin=3cm]{geometry}

\title{Decomposing the Government-Private School \\ Performance Differential: \\ Village Ethnic Politics and School Sorting}
\author{Nicholas Eubank\footnote{Post-Doctoral Fellow in Political Science. \href{mailto:nicholaseubank@stanford.edu}{nicholaseubank@stanford.edu}. This project would not have been possible without the exceptional support numerous parties, including Jishnu Das, Tahir Andrabi, Alex Lee, Kate Casey, Neil Malhotra, Meredith Startz, and Paul Novosad.  Code and replication directions can be found at \href{http://www.github.com/nickeubank/pk_schooling}{\url{www.github.com/nickeubank/pk_schooling}}} \\ \emph{Stanford University}}

\date{\today}

% This is the beginning of a real document!

\begin{document}
\maketitle
\begin{center}
\vspace{1.5cm}

\textbf{PRELIMINARY DRAFT \\ PLEASE DO NOT CITE WITHOUT AUTHOR’S PERMISSION}

{\Large \color{blue}\href{http://www.nickeubank.com/eubank_schoolsorting/}{Click here to download the most recent version of this paper.}\color{black}}
\vspace{1.5cm}\\

\end{center}

\vspace{0.5cm}

\begin{abstract}
The emergence of rural, secular, affordable private schools across South Asia is one of the most promising recent developments in the education sector. Yet whether private schools provide superior educations remains unclear. Observational studies consistently show private school students outperform government students even when controlling for demographic characteristics and some unobservable heterogeneity using Value-Added models. Nevertheless, it remains unclear whether this is because (a) private schools provide students with a better education, or (b) students attending private schools are more academically inclined in unobservable ways. Using data from the Learning and Educational Attainment in Punjab Schools (LEAPS) survey, this paper leverages variation in sorting on academic potential caused by village caste politics to isolate the component of private school performance caused by sorting rather than superior teaching. It concludes that even the most sophisticated observational techniques -- lagged Value-Added models -- overstate private school quality by at least half. 
\end{abstract}

\thispagestyle{empty}



\pagebreak

\setcounter{page}{1}

\section{Introduction}\label{pk_intro}

	\subfile{part_introduction.tex}

\section{Study Context}\label{pk_context}

	\subfile{part_context.tex}

\section{Caste Politics and School Sorting}\label{pk_sorting} % Frac (fold)

	\subfile{part_sorting.tex}


\section{School Sorting and Test Scores}\label{pk_scores}

	\subfile{part_scores.tex}


\section{Alternate Explanations}\label{pk_alternatives}

	\subfile{part_alternatives.tex}

\section{Discussion and Policy Implications}\label{pk_conclusion}

	\subfile{part_conclusion.tex}

\pagebreak

	\bibliography{/Users/Nick/Documents/my_library}
	\bibliographystyle{apalike}

\appendix

\section{Value-Added Test Scores}\label{appendix_valueadded}

	\subfile{appendix_valueadded.tex}
\clearpage

\section{Biraderi Classification}\label{appendix_classification}

	\subfile{appendix_classification.tex}



\end{document}
