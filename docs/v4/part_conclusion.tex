\documentclass[Eubank_pk_ethnic_sorting.tex]{subfiles}


\begin{document}


Private schools represent a radical departure from the educational status quo in nearly every way. Where government schools hire teachers with college educations, emphasize teacher training, and pay high wages, private schools hire secondary-educated women from the local community for a fraction of the cost, invest nothing in teacher training, and appear to reward performance. If government schools were to emulate their business model, tax payers would save millions. But would educational outcomes improve? 

The results presented here cannot conclusively answer this question. Even in the most heterogeneous villages, private schools still outperform government schools (just by dramatically less than in homogeneous villages). But these results do suggest at least two major reasons for caution on the part of policy-makers thinking about the implications of private school growth, and whether recent calls for the widespread distribution of private school vouchers (e.g. \cite{Chakrabarti:2008vc}, \cite{Kelkar:2006tq}, and \cite{Panagariya:2008wi}) are good policy.

First, this analysis shows that even with a lagged-value-added specification, parental education and wealth controls, and panel data, it is still easy to vastly over-estimate the contributions of private schools to learning using observational data. As such, this study illustrates the need for researchers and policy-makers to maintain a health degree of skepticism when reading observational studies that claim to separate sorting from school quality.

Second, while this analysis only shows that 50\% of the private school premium can be explained by sorting, it is important to remember that this represents a lower-bound on the contributions of sorting. The difference between homogeneous and heterogeneous villages is not ``sorting on intelligence'' and ``no sorting on intelligence,'' but rather ``sorting on intelligence'' and ``\emph{less} sorting on intelligence.'' As such, sorting is still likely contributing to the private school premium in heterogeneous villages. Indeed, perceived intelligence remains an important determinant of school choice even in highly fractionalized villages. 

In light of persistent poor performance among government schools, the hope that private schools will reform the South Asian education sector is understandable, and may yet prove to be well founded. But as shown here the superiority of private school is not self-evident, and the government would do well to gather more evidence before embracing private schools as a substitute for government schools. 

\end{document}

