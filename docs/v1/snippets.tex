

Advocates of private schools argue that not only are observational studies able to control for many factors, but there is also evidence to explain \emph{why} private schools outperform government schools. In particular, they point out that private schools appear to address the biggest problem in government schools: low effort. High absenteeism and low accountability in government schools has been well documented \citep{Muralidharan:2008tb, Chaudhury:2006vp}, but appears less prevalent in private schools. The reason, many argue, is that in private schools good teachers are better paid, and poor teachers are let go\citep{Andrabi:2007we}. This line of reasoning is also buoyed by a growing body of literature that suggests that what matters for success is not the availability of educational ``inputs'' (like qualified, well paid teachers or good facilities), but incentive schemes that reward effort on the behalf of teachers.\citep{Hanushek:1997tt,Hanushek:2003hz,Banerjee:2007wx}.\footnote{This reasoning is not entirely supported by the evidence. For example, while private schools generally outperform public schools in the US and many have attributed that to better incentive structures, private-run government charter schools (which presumably have a similar incentive system but do not necessarily have the saw draw for educationally minded families) have not faired so well.\citep{Fuller:2002td} And while a number of studies of randomly-assigned private-school vouchers suggest that the private school effect may be real, even these studies have been complicated by a number of problems which make inference difficult.} 


This paper evaluates these two theories -- the ``incentive'' theory and the ``sorting'' theory -- on the basis of their ability to explain a novel empirical regularity. Using data from 112 villages in rural Pakistan, this paper shows that private school dominance varies with village fractionalization. In particular, it shows that as the level of caste fractionalization in a village increases, the degree to which private schools outperform government schools declines by 50\%. 

As this paper will argue, this finding appears to be incompatible with the ``incentive'' story as there is no evidence that the way teachers are paid in either government or private schools varies between high and low fractionalization villages. By contrast, there is relatively strong evidence to suggest that the way students ``sort'' between schools does vary with ethnic fractionalization, and that this change in sorting is consistent with the convergence in test scores seen in the data. 



-------

\subsubsection{WHERE?}

The preceding section showed the incompatibility of the ``incentive'' story with the convergence of public and private test scores, suggesting that this phenomenon may instead be the result of changes in how students ``sort'' across schools.

The evidence appears to support this conclusion. If it were the case that convergence were driven by government school improvement or private school decline, one would expect that to be reflected in some change in overall scores. But as shown in Table~\ref{kidsnointeract}, this is not the case. Overall test scores are essentially flat across villages -- English scores are slightly higher in more fractionalized villages in Column 1, but the magnitude of this difference is relatively small, and once more demographic controls are added in Column 2 this effect disappears. No relationship exists for other subjects. Public scores increase and private scores decline with fractionalization, but those changes are almost perfectly offsetting.  

\begin{sidewaystable}[htbp]\centering
\def\sym#1{\ifmmode^{#1}\else\(^{#1}\)\fi}
\caption{Child Test Scores and Fractionalization \label{kidsnointeract}}
\begin{tabular}{l*{6}{c}}
\toprule
                &\multicolumn{2}{c}{English}&\multicolumn{2}{c}{Urdu} &\multicolumn{2}{c}{Math} \\\cmidrule(lr){2-3}\cmidrule(lr){4-5}\cmidrule(lr){6-7}
                &\multicolumn{1}{c}{(1)}&\multicolumn{1}{c}{(2)}&\multicolumn{1}{c}{(3)}&\multicolumn{1}{c}{(4)}&\multicolumn{1}{c}{(5)}&\multicolumn{1}{c}{(6)}\\
                &\multicolumn{1}{c}{}&\multicolumn{1}{c}{}&\multicolumn{1}{c}{}&\multicolumn{1}{c}{}&\multicolumn{1}{c}{}&\multicolumn{1}{c}{}\\
\midrule
Private School  &     0.31***&     0.29***&     0.14***&     0.14***&     0.11***&    0.088** \\
                &  (0.028)   &  (0.028)   &  (0.025)   &  (0.024)   &  (0.035)   &  (0.034)   \\
Biraderi Fractionalization&     0.14*  &     0.10   &    0.080   &    0.057   &     0.13   &     0.12   \\
                &  (0.075)   &  (0.073)   &  (0.066)   &  (0.064)   &  (0.094)   &  (0.089)   \\
Lagged English Scores&     0.40***&     0.39***&     0.16***&     0.14***&     0.17***&     0.16***\\
                &  (0.018)   &  (0.018)   &  (0.012)   &  (0.012)   &  (0.016)   &  (0.016)   \\
Lagged Math Scores&    0.067***&    0.070***&     0.12***&     0.12***&     0.39***&     0.40***\\
                & (0.0082)   & (0.0083)   & (0.0083)   & (0.0086)   &  (0.013)   &  (0.014)   \\
Lagged Urdu Scores&     0.15***&     0.15***&     0.39***&     0.40***&     0.23***&     0.22***\\
                &  (0.011)   &  (0.012)   &  (0.011)   &  (0.012)   &  (0.013)   &  (0.013)   \\
Village: Pct Adults Literate&  0.00058   &  0.00017   & -0.00027   & -0.00060   &  0.00031   &  0.00025   \\
                &(0.00097)   &(0.00098)   &(0.00086)   &(0.00085)   & (0.0014)   & (0.0013)   \\
Log Number of Households&    0.017   &    0.019   &    0.012   &    0.015   &   0.0083   &    0.011   \\
                &  (0.013)   &  (0.013)   &  (0.015)   &  (0.014)   &  (0.019)   &  (0.017)   \\
Village Land Gini&  -0.0043   &    0.036   &   0.0079   &    0.057   &    -0.29** &    -0.26*  \\
                &   (0.13)   &   (0.12)   &   (0.11)   &  (0.098)   &   (0.13)   &   (0.13)   \\
Child's Wealth Index&            &    0.015***&            &   0.0068** &            &    0.016***\\
                &            & (0.0035)   &            & (0.0030)   &            & (0.0042)   \\
Educated Parent &            &    0.053***&            &    0.049***&            &    0.043***\\
                &            &  (0.013)   &            &  (0.011)   &            &  (0.014)   \\
Constant        &     0.45   &     0.64** &     0.68***&     0.81***&     0.34   &     0.62*  \\
                &   (0.28)   &   (0.25)   &   (0.25)   &   (0.28)   &   (0.37)   &   (0.35)   \\
District Fixed Effects&      Yes   &      Yes   &      Yes   &      Yes   &      Yes   &      Yes   \\
\midrule
Observations    &    37147   &    26141   &    37147   &    26141   &    37147   &    26141   \\
\bottomrule
\multicolumn{7}{l}{\footnotesize Standard errors in parentheses}\\
\multicolumn{7}{l}{\footnotesize * \(p<0.10\), ** \(p<0.05\), *** \(p<0.01\)}\\
\multicolumn{7}{l}{\footnotesize Controls for age, age squared, gender, and class omitted from table. Standard errors clustered at village level.}\\
\end{tabular}
\end{sidewaystable}


The most obvious explanation for this pattern is that intelligent children are more likely to attend private schools in homogenous villages than they are in fractionalized villages. But why?

This section presents evidence that in all villages, children who are perceived to be more intelligent are more likely to be sent to private schools (presumably because the education of a gifted child is perceived to be a better investment by parents). But in more heterogenous villages, school choice is also influenced by a desire to send children to ``caste appropriate'' schools where they will be surrounded by children of a similar status. As a result, ``high status'' families are more likely to send their children to private schools \emph{regardless of their academic potential}. Thus high-status but low-achieving children who would be left in a government school in a homogenous village end up in private schools, dragging down test scores, and vice-versa for low status children (who are either driven out by social pressure or higher prices). This weakens the degree to which children are sorting on academic potential, leading to convergence of public and private schools.

Note that this argument is dependent upon some basic assumptions about the distribution of talent across castes. In particular, it requires that residual talent -- talent that cannot be explained by things like parental education and wealth -- be either relatively equally distributed across the different social strata, or be distributed slightly in favor of lower status \emph{biraderis}. If not, and even the least talented ``high status'' students were more talented than the most talented ``low status'' students, then this type of sorting could result in \emph{divergence}, rather than \emph{convergence}, of test scores. As shown in Table~\ref{castesarentdumb}, however, there is no evidence that those from higher social status \emph{biraderis} have higher residual talent than those from low status \emph{biraderis}.

\begin{sidewaystable}[htbp]\centering
\def\sym#1{\ifmmode^{#1}\else\(^{#1}\)\fi}
\caption{Child Social Status and Residual Talent\label{castesarentdumb}}
\begin{tabular}{l*{9}{c}}
\toprule
                &\multicolumn{3}{c}{English}           &\multicolumn{3}{c}{Urdu}              &\multicolumn{3}{c}{Math}              \\\cmidrule(lr){2-4}\cmidrule(lr){5-7}\cmidrule(lr){8-10}
                &\multicolumn{1}{c}{(1)}&\multicolumn{1}{c}{(2)}&\multicolumn{1}{c}{(3)}&\multicolumn{1}{c}{(4)}&\multicolumn{1}{c}{(5)}&\multicolumn{1}{c}{(6)}&\multicolumn{1}{c}{(7)}&\multicolumn{1}{c}{(8)}&\multicolumn{1}{c}{(9)}\\
                &\multicolumn{1}{c}{}&\multicolumn{1}{c}{}&\multicolumn{1}{c}{}&\multicolumn{1}{c}{}&\multicolumn{1}{c}{}&\multicolumn{1}{c}{}&\multicolumn{1}{c}{}&\multicolumn{1}{c}{}&\multicolumn{1}{c}{}\\
\midrule
High Status Zaat&   -0.040   &   -0.072   &    -0.10** &   -0.037   &    -0.10   &   -0.081   &   0.0014   &   -0.025   &    0.036   \\
                &  (-0.89)   &  (-1.03)   &  (-2.27)   &  (-0.63)   &  (-1.42)   &  (-1.57)   &   (0.02)   &  (-0.29)   &   (0.54)   \\
Private School  &     0.31** &     0.21   &     0.21*  &     0.28** &     0.26*  &     0.14   &   -0.034   &   -0.060   &    -0.11   \\
                &   (2.62)   &   (1.65)   &   (1.80)   &   (2.34)   &   (1.91)   &   (1.13)   &  (-0.31)   &  (-0.44)   &  (-0.75)   \\
Fractionalization * Private&   -0.033   &    0.089   &    0.041   &    -0.21   &    -0.19   &   -0.066   &     0.27   &     0.33   &     0.29   \\
                &  (-0.17)   &   (0.42)   &   (0.21)   &  (-1.14)   &  (-0.85)   &  (-0.33)   &   (1.44)   &   (1.37)   &   (1.21)   \\
Lagged English Scores&     0.32***&     0.31***&     0.41***&     0.16***&     0.16***&     0.16***&     0.18***&     0.19***&     0.18***\\
                &   (7.60)   &   (6.82)   &  (10.17)   &   (5.10)   &   (4.46)   &   (5.29)   &   (4.11)   &   (4.10)   &   (4.48)   \\
Lagged Math Scores&    0.060** &    0.048   &    0.045   &     0.11***&    0.076*  &    0.088** &     0.30***&     0.27***&     0.37***\\
                &   (2.01)   &   (1.36)   &   (1.40)   &   (2.76)   &   (1.81)   &   (2.19)   &   (6.01)   &   (5.35)   &   (7.63)   \\
Lagged Urdu Scores&     0.17***&     0.19***&     0.17***&     0.34***&     0.35***&     0.41***&     0.25***&     0.26***&     0.26***\\
                &   (4.53)   &   (4.08)   &   (4.07)   &   (8.74)   &   (7.63)   &   (8.85)   &   (5.39)   &   (4.95)   &   (4.92)   \\
Child's Wealth Index&            &  -0.0067   &   -0.011   &            &   0.0020   &   0.0010   &            &   -0.014   &  -0.0094   \\
                &            &  (-0.50)   &  (-0.93)   &            &   (0.19)   &   (0.11)   &            &  (-0.82)   &  (-0.67)   \\
Educated Parent &            &     0.14***&     0.14***&            &     0.14***&     0.14***&            &     0.17***&     0.18***\\
                &            &   (3.19)   &   (3.55)   &            &   (3.27)   &   (3.91)   &            &   (3.04)   &   (3.85)   \\
Mauza Biraderi Fractionalization (from top 24 codes)&            &            &   -0.018   &            &            &   -0.072   &            &            &   -0.055   \\
                &            &            &  (-0.17)   &            &            &  (-0.56)   &            &            &  (-0.32)   \\
Village: Pct Adults Literate&            &            &  0.00038   &            &            &  -0.0032** &            &            &  -0.0013   \\
                &            &            &   (0.21)   &            &            &  (-2.07)   &            &            &  (-0.49)   \\
Log Number of Households&            &            &    0.020   &            &            &    0.026   &            &            &    0.051   \\
                &            &            &   (0.56)   &            &            &   (0.77)   &            &            &   (0.84)   \\
Village Land Gini&            &            &    0.026   &            &            &     0.11   &            &            &    -0.15   \\
                &            &            &   (0.14)   &            &            &   (0.52)   &            &            &  (-0.45)   \\
Constant        &     0.57   &     0.71   &     0.75   &     1.18*  &     2.35***&     2.22***&     1.57** &     2.83***&     2.18** \\
                &   (1.02)   &   (1.00)   &   (1.26)   &   (1.85)   &   (3.18)   &   (3.31)   &   (2.12)   &   (3.07)   &   (2.31)   \\
Village Fixed Effects&      Yes   &      Yes   &       No   &      Yes   &      Yes   &       No   &      Yes   &      Yes   &       No   \\
District Fixed Effects&       No   &       No   &      Yes   &       No   &       No   &      Yes   &       No   &       No   &      Yes   \\
\midrule
Observations    &     1790   &     1323   &     1323   &     1790   &     1323   &     1323   &     1790   &     1323   &     1323   \\
\bottomrule
\multicolumn{10}{l}{\footnotesize \textit{t} statistics in parentheses}\\
\multicolumn{10}{l}{\footnotesize * p<0.10, ** p<0.05, *** p<0.01}\\
\end{tabular}
\end{sidewaystable}



\subsection{Limitations of Sorting Evidence}\label{}

Some of the results predicted by the sorting story do not appear in the data, however. For example, this story suggests that the importance of perceived intelligence in school choice should decline with village fractionalization. Table~\ref{hhselectioninteraction} below employs the same specification used in Table~\ref{hhselection} to show that parents are more likely to send children they perceive as intelligent to private school, but with the addition of an interaction term on child intelligence. As shown in the table, there is no evidence of a change in the coefficient on perceived intelligence with village fractionalization for either high or low status families.

Interpreting this result is made somewhat difficult by the structure of the LEAPS data. Ideally, these regressions would be run for the full 30,000 children surveyed, but household data is only available for 4,212 of these children (who come from, at most, 16 households per village). As such, there is reason to believe that this ``non-result'' may reflect the limited quality of the data rather than the accuracy of the sorting story. Nevertheless, the lack of a finding in this setting must be considered when weighing the strength of the evidence here. 

\begin{table}[htbp]\centering
\def\sym#1{\ifmmode^{#1}\else\(^{#1}\)\fi}
\caption{School Choice and Child Intelligence\label{hhselectioninteraction}}
\begin{tabular}{l*{6}{c}}
\hline\hline
                &\multicolumn{2}{c}{All}  &\multicolumn{2}{c}{High Status}&\multicolumn{2}{c}{Low Status}\\\cmidrule(lr){2-3}\cmidrule(lr){4-5}\cmidrule(lr){6-7}
                &\multicolumn{1}{c}{(1)}&\multicolumn{1}{c}{(2)}&\multicolumn{1}{c}{(3)}&\multicolumn{1}{c}{(4)}&\multicolumn{1}{c}{(5)}&\multicolumn{1}{c}{(6)}\\
                &\multicolumn{1}{c}{}&\multicolumn{1}{c}{}&\multicolumn{1}{c}{}&\multicolumn{1}{c}{}&\multicolumn{1}{c}{}&\multicolumn{1}{c}{}\\
\hline
\specialcell{Mom: Child Above\\Avg Intelligence}&    0.056   &    0.063   &     0.15   &    0.085   &    0.075   &     0.28   \\
                &   (0.66)   &   (1.01)   &   (1.53)   &   (1.22)   &   (0.41)   &   (1.25)   \\
Biraderi Fractionalization&   -0.022   &     0.19   &     0.90***&     0.18   &            &     0.45** \\
                &  (-0.55)   &   (1.00)   &  (22.83)   &   (0.76)   &            &   (2.64)   \\
\specialcell{Child Above Avg *\\Fractionalization}&   0.0029   &   -0.031   &    -0.14   &   -0.067   &   -0.067   &    -0.35   \\
                &   (0.02)   &  (-0.35)   &  (-0.96)   &  (-0.64)   &  (-0.26)   &  (-1.19)   \\
Mom Has Some Schooling&    0.081   &   -0.033   &    0.088   &   -0.013   &     0.12   &   -0.100   \\
                &   (1.51)   &  (-0.28)   &   (1.40)   &  (-0.08)   &   (1.05)   &  (-1.08)   \\
Mom Has Some Schooling&    0.084***&    0.083   &    0.094***&    0.023   &   -0.094   &     0.22***\\
                &   (3.23)   &   (0.71)   &   (2.78)   &   (0.13)   &  (-1.09)   &   (2.80)   \\
Log Month Expenditure&    0.043*  &   -0.010   &    0.036   &     0.39***&     0.11   &   -0.058** \\
                &   (1.79)   &  (-1.19)   &   (1.25)   &  (15.17)   &   (1.48)   &  (-2.18)   \\
Age             &   -0.021***&   -0.017***&   -0.022***&   -0.019***&   -0.097   &   -0.058   \\
                &  (-3.76)   &  (-3.26)   &  (-4.68)   &  (-3.20)   &  (-1.11)   &  (-0.53)   \\
Age Squared     &  0.00025*  &  0.00017   &  0.00025*  &  0.00022** &   0.0035   &   0.0023   \\
                &   (1.78)   &   (1.62)   &   (1.93)   &   (2.01)   &   (0.77)   &   (0.40)   \\
Female          &    0.029   &  -0.0013   &   -0.017   &   -0.010   &     0.14*  &    0.090   \\
                &   (1.27)   &  (-0.05)   &  (-0.77)   &  (-0.41)   &   (1.81)   &   (0.92)   \\
Constant        &    -0.23   &    0.087   &    -0.38   &    -3.12***&    -0.40   &     0.40   \\
                &  (-1.13)   &   (0.94)   &  (-1.62)   & (-13.72)   &  (-0.50)   &   (0.69)   \\
Village Fixed Effects&      Yes   &       No   &      Yes   &       No   &      Yes   &       No   \\
Household Fixed Effects&       No   &      Yes   &       No   &      Yes   &       No   &      Yes   \\
\hline
Observations    &     3426   &     3426   &     2212   &     2212   &      440   &      440   \\
\hline\hline
\multicolumn{7}{l}{\footnotesize \textit{t} statistics in parentheses}\\
\multicolumn{7}{l}{\footnotesize * p<0.10, ** p<0.05, *** p<0.01}\\
\end{tabular}
\end{table}





\footnote{This reasoning is not entirely supported by the evidence. For example, while private schools generally outperform public schools in the US and many have attributed that to better incentive structures, private-run government charter schools (which presumably have a similar incentive system but do not necessarily have the saw draw for educationally minded families) have not faired so well.\citep{Fuller:2002td} And while a number of studies of randomly-assigned private-school vouchers suggest that the private school effect may be real, even these studies have been complicated by a number of problems which make inference difficult.} 



