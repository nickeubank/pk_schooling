\documentclass[Eubank_pk_ethnic_sorting.tex]{subfiles}


\begin{document}

The rapid rise of affordable and apparently high quality private schools in South Asian rural communities is one of the most exciting developments in the education sector in decades. Private schools account for an ever rising share of children attending these schools -- in 2005, 33\% of Pakistani primary school students and 20-24\% of Indian rural primary school students attended a private school, and the students in these private schools consistently outperform their government school counterparts, even when controlling for observable student characteristics \citep{Jimenez:1991wa, Jimenez:1995vg, Pratham:2005vw, Andrabi:2011hl, Desai:2009ty, Tooley:2003vf, Alderman:2003we, Alderman:2001wk}. This has given rise to the hope that private schools may someday circumvent reform-resistant government schools and finally deliver quality educations to the hundreds of millions of children in the region.

Despite the promise of these developments, however, the true significance of the rapid emergence of private schools hinges critically on the question of whether these private schools are actually delivering superior educations, or whether they just attract students who are more academically inclined or come from families that prioritize educational attainment. 

This paper leverages variation across villages in the factors that drive school choice to provide new insight into this question. In particular, it takes advantage of the fact that school choice in rural Pakistan is motivated by different factors in caste-homogeneous and caste-heterogeneous villages. This paper shows that in caste-diverse villages, school choice is driven by social considerations -- high-status families send their children to private schools to keep them in homogeneous social settings. In caste-homogeneous villages where caste sorting is unnecessary, by contrast, school choice is driven by parental perceptions of academic potential -- families send their more academically gifted children to private schools. 

Because school choice is primarily based on academic potential in homogeneous villages and not in heterogeneous villages, differences in the government-private test score gap across these types of villages can be interpreted as a lower-bound on the contribution of student sorting to the test score gap.\footnote{There is likely still \emph{some} sorting on potential in heterogeneous villages, thus this result can only be interpreted as a lower-bound.} Using this empirical framework, this paper finds that at least 1/2 of the test score gap between government and private schools that persists after controlling for both observable and many unobservable student characteristics can be attributed to student sorting, suggesting that while private schools may still be outperforming government schools, the magnitude of this differential is likely grossly overstated in non-experimental work.\footnote{Of course efforts have been made to address the possibility of student ``sorting.'' Randomizing school assignments is untenable, but in two major cases private school vouchers have been randomly assigned. Even these randomizations have proven problematic, however. \cite{Angrist:2002up} examines a voucher lottery system in Colombia and finds a small positive effect of vouchers, but inference is clouded by the fact that voucher students who performed poorly were at risk of losing their vouchers, making it impossible to separate this incentive effect from the private school effect. And several studies have been conduced of a voucher system in Chile, but as \cite{Bellei:2008uu} notes, the slight private-school advantage these studies show may be down to the fact that private school admissions are selective and poorly performing students can be expelled from private schools, making it difficult to disentangle selectivity from school effects.}

It is difficult to overstate the potential importance of the answer to this question for education policy in the developing world. Not only do private schools constitute a substantial portion of current enrollments in South Asia, but enrollment is also growing explosively. From 2000 to 2005 in rural Pakistan, for example,``the number of private schools in Pakistan increased from 32,000 to 47,000.'' \citep[p. vi]{Andrabi:2007we}, and evidence suggests this growth continues today. Moreover, private schools deliver educations at a fraction of the cost of government schools by not requiring formal teacher training and by hiring local secondary-educated women as teachers rather than college-educated teachers who have to move to the villages where they teach \citep{Andrabi:2007we}. Thus the question of whether these lower-cost educations are of similar quality to government school educations also speaks to the importance of different teacher training and employment practices.

The findings of this paper are unlikely to put to rest the debate over whether private schools are superior to government schools. Indeed, private school students continue to outperform government school students even in the most homogeneous villages, just by a dramatically smaller margin. But even after using some of the most sophisticated econometric methods available, this paper still establishes (as a  ``lower bound'') that sorting explains a very large portion of the public-private test score gap. This should give analysts pause when examining other empirical results that claim to fully control for sorting. 

This paper is organized out as follows: Section~\ref{context} provides an overview of the rural Pakistan context from which data for this analysis is drawn. Section~\ref{sorting} details how the determinants of school choice vary with village caste heterogeneity. Section~\ref{scores} then shows how the government-private test score gap varies with caste heterogeneity, and presents evidence this is due to differences in student sorting. Section~\ref{alternates} then examines and rules out a number of alternative possible explanations for this empirical regularity. And finally, Section~\ref{conclusion} discusses the strengths and weaknesses of these findings and their interpretation. 

\end{document}